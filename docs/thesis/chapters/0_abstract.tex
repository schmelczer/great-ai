\begin{abstract}

\absdiv{Background}
Despite its long-standing history, artificial intelligence (AI) has only recently started enjoying widespread industry awareness and adoption, partly thanks to the prevalence of libraries that accessibly expose state-of-the-art models. However, the transition from prototypes to production-ready AI applications is still a source of struggle across the industry. Even though professionals already have access to frameworks for deploying AI, case studies and developer surveys have found that many deployments do not follow best practices.

\absdiv{Objective}
This thesis investigates the causes of and presents a possible solution to the asymmetry between the adoption of libraries for \textit{applying} and those for \textit{deploying} AI. The potential solution is validated through designing a software framework called \textit{GreatAI}, which aims to facilitate \underline{G}eneral \underline{R}obust \underline{E}nd-to-end \underline{A}utomated \underline{T}rustworthy deployments while attempting to overcome the practical drawbacks of earlier similar tools, e.g., \textit{Seldon Core}, \textit{AWS SageMaker}, and \textit{TensorFlow Extended}.

\absdiv{Methods}
\textit{GreatAI} serves as a proxy for exploring the proposed design decisions; moreover, its initial focus is limited to the domain of natural language processing (NLP). Its design is created by applying the principles of design science methodology through iteratively shaping it in two case studies of a commercial NLP pipeline. Subsequently, interviews are conducted with ten practitioners to assess its applicability and generalisability.

\absdiv{Results}
\textit{GreatAI} helps implement 33 best practices through an accessible interface. These target the transition between the prototype and production phases of the AI development lifecycle. Feedback from professional data scientists and software engineers showed that ease of use and functionality are equally important in deciding to adopt deployment technologies, and the proposed framework was rated positively in both dimensions.

\absdiv{Conclusions}
Increasing the overall maturity of industrial AI deployments by devising APIs with ease of adoption in mind is proved to be feasible. While \textit{GreatAI} mainly focuses on NLP, the results show that the development and deployment of trustworthy AI services, in general, can be assisted by frameworks prioritising easy adoption while still streamlining the implementation of various best practices.

\end{abstract}
