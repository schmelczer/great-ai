\chapter{Conclusion}  \label{chapter:conclusion}

todo

\textit{GreatAI} may have the potential to bridge the gap between data science and software engineering. Stemming from the bidirectional nature of bridges, we can look at the framework from two perspectives: for professionals closer to the field of data science, it provides an automatic scaffolding of software facilities that are required for deploying, monitoring, and iterating on their models. For software engineers, it highlights the necessary steps required for robust and improvable deployments --- while at the same time --- saves them from the menial work of implementing these constructs manually. While most importantly, it proves that increasing the adoption rate of AI/ML deployment best practices is viable by designing narrower and deeper APIs.

Good deployments benefit all of us. Continued research into the means of good deployments remains crucial. However, next to that --- as the presented results show --- better deployments can be also achieved by facilitating the \textit{transition} step of the AI lifecycle. Having automated implementations, even if for just simpler best practices, leaves professionals more time to tackle other deployment challenges and less opportunities to miss crucial steps. Overall, resulting in more implemented practices, hence, robust and trustworthy production software.

\section{Concluding remarks}
